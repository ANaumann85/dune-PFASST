\documentclass{scrreprt}%[12pt,titlepage,a4paper]%{scrartcl} 
\usepackage[fleqn]{amsmath}
\usepackage[ngerman]{babel}
\usepackage[utf8]{inputenc}
\usepackage{graphicx}
\usepackage{subfigure} 
\usepackage{url}
\usepackage{color}
\usepackage{listings}
\usepackage{float}
\usepackage{wrapfig}
\usepackage{amssymb}
\usepackage[T1]{fontenc}
\usepackage{lmodern} 
\usepackage{amsthm}
\usepackage[onehalfspacing]{setspace}
\usepackage{pgfplots}
%\usepackage{shadethm}
%\newshadetheorem{thms}{Definition}[chapter]

%jochen
%\usepackage{graphics}
%\usepackage[colorlinks=true]{hyperref}
%\usepackage{hyperref}

%\usepackage{placeins}
%\usepackage[left=1.5cm,right=1.5cm,top=3.5cm,bottom=3cm]{geometry}
%\usetikzlibrary{shapes,arrows,calc,external,matrix,positioning}
%jochen

\pgfplotsset{
   01_temp/.style={
      legend columns=2,
      legend style={
		 at={(0.02,0.98)},anchor=north west,font=\footnotesize,rounded corners=2pt,         
         %font=\scriptsize,
         %legend pos=north east,
         %draw=none,
         %/tikz/column 2/.style={
         %   column sep=5pt,
         %}%,
         %legend style={ }
      },
      xlabel={Zeit $t$ [h]},
      x tick style={
        color=black,
        thin
      },      
      ylabel={Knotentemperatur $T$ [$^\circ$C]},
      y tick style={
        color=black,
        thin
      },
      height=9cm,
      width=15cm,
      grid=major,
      grid style={
         solid,
         ultra thin,
         gray
      },
      /pgf/number format/.cd,
      use comma,
      set thousands separator={},
   }
}



	
\begin{document}



mlsdc

\begin{figure}[H]
%\begin{minipage}{0.2\linewidth}
  \begin{center}
  	  \begin{tikzpicture}[scale=1]
		  \begin{loglogaxis}[01_temp, ymax=1e-1,ymin=1e-12,xlabel={$\Delta t$},ylabel={Fehler $e_{\Delta t}$},grid=major]	
		  	\addplot table[x index = 1, y index = 2] {../../../build-cmake/src/examples/FE_Newton_linear/solution_mlsdc/200.dat};
		  	\addplot table[x index = 1, y index = 2] {../../../build-cmake/src/examples/FE_Newton_linear/solution_mlsdc/400.dat};
		   \legend{}
		  \end{loglogaxis}
	  \end{tikzpicture}
  \end{center}
%\end{minipage}
%\caption{Konvergenzordnung des hyperbolischen Splittings für $\varepsilon=1$}
%\label{konv}
\end{figure}










\end{document}